%        File: manual.tex
%     Created: Thu May 23 08:00 AM 2019 E
%
\documentclass[letterpaper,11pt]{article}
\usepackage[leqno]{amsmath}
\usepackage{mathptmx,courier}
\usepackage[scaled]{helvet}
\usepackage[inter-unit-product=\ensuremath{{}\cdot{}},per-mode=symbol,separate-uncertainty,binary-units]{siunitx}
\usepackage[nohead,letterpaper,lmargin=0.75in,rmargin=0.75in,tmargin=0.75in,bmargin=0.75in]{geometry}
\usepackage{paralist}
\usepackage[pdftex]{graphicx}
\usepackage[hyperfigures=false,bookmarks=false,pdftex,colorlinks,linkcolor=black,urlcolor=blue]{hyperref}
\usepackage{fancyhdr,lastpage}
\usepackage[cachedir=/home/kjh016/tmp/mintedcache]{minted}
\usepackage{fancyvrb}
\newcommand\userinput[1]{\textbf{#1}}
\usepackage{tcolorbox}

\graphicspath{{graphics/}}

\ifpdf\pdfinfo{
  /Title (Firefly Simulator User's Manual) %FIXME
  /Author (K. J. Hass, Bucknell University) %FIXME
/Keywords () }
\fi

\pagestyle{fancy}
\fancyhf{}
\fancyfoot[C]{Page \thepage\ of \pageref{LastPage}}
\renewcommand\headrulewidth{0pt}

\begin{document}
\Large
\centerline{Firefly Simulator User's Manual}
\normalsize
\subsection*{Connections}
\subsubsection*{Power/USB}
\subsubsection*{Physical LEDs}

A \textit{physical LED} consists of a small circuit board connected to a cable
that can be connected to the simulator. An actual LED is mounted on the circuit
board, as well as other components that are used to regulate and limit the
electrical current through the LED.

Up to six physical LEDs can be simultaneously connected to the simulator, by
plugging their cables into the six \textit{LED channel} connectors on the side
of the simulator.

It is the user's responsibility to mark each physical LED with a unique
identifier and maintain records of the optical characteristics of that LED.
It is also the user's responsibility to track which physical LED is connected
to each LED channel.

\subsubsection*{Abort Button}

\subsection*{Communications}

The firefly simulator communicates with a host computer using a serial
communications interface, which is called a \texttt{COM} port in Windows or a
\texttt{/dev/tty} in Linux and MacOS. This interface can be used to configure
the simulator and to log its operation.

The simulator has been programmed to operate at 9600 baud with 8 data bits,
no parity, 1 stop bit, and no flow control.

The user can send commands to the simulator using a terminal simulator program
on the host computer, and the simulator's responses will also appear on the
terminal. The command messages sent by the host computer are simple strings of
text. Each command begins with one or two \textbf{capital} letters, followed
by some number of \textbf{fields} or \textbf{parameters} for the command. The
different fields of a message are separated by a \textbf{comma}. Each command
must be terminated with a carriage return and/or linefeed, which are typically
added just by pressing the ``\texttt{Enter}'' key.

\subsection*{Virtual LEDs}

The simulator software has no information about what, if any, physical LEDs are
connected to its LED channels. Instead, the simulator software is configured to
control \textit{virtual LEDs}. A virtual LED is defined by assigning to it a
physical LED channel number as well as a maximum brightness value (0 to 100).
Note that the maximum brightness level is treated as a percentage of the
maximum current available to the physical LED. For example, if a particular
physical LED circuit board has been designed to limit the LED current to
\SI{20}{\milli\ampere}, and then setting the maximum brightness of a virtual
LED to \texttt{50}, will result in an average LED current of
\SI{10}{\milli\ampere} when the LED is ``on''.

Each virtual LED has a unique \textit{LED number}, which is simply an integer
used to identify a particular virtual LED so it can be used later. The
simulator can typically store up to 16 distinct virtual LED definitions.

Note that a given LED channel (i.e.\ a given physical LED) can be used in more
than one virtual LED definition. This allows the same physical LED to be used
at different brightness levels.

The command message used to configure a virtual LED begins with the letter `L'
and has three fields: the virtual LED number, the LED channel number, and the
maximum brightness level. The fields must be integer values. For example, the
commands shown below will first configure virtual LED \#7 to use the physical
LED on channel 3 with a maximum brightness equivalent to 80\% of its maximum
possible current. Virtual LED \#1 is then configured to use channel 2 with
100\% of its maximum current. If the format of the command is correct and the
fields have appropriate values, then the simulator will respond with
``\texttt{LED Configured}''. 

\begin{tcolorbox}
\begin{Verbatim}[commandchars=\\\{\}]
\userinput{L,7,3,80}
LED Configured
\userinput{L,1,2,100}
LED Configured
\end{Verbatim}
\end{tcolorbox}

Note that once configured a virtual LED can be modified simply by rewriting
its definition. Defining the same virtual LED number multiple times does not
produce an error but only the last definition will be saved.

The \texttt{DL} (Dump LEDs) command will list all of the virtual LEDs that
have been configured in the simulator. There are no additional fields for this
command. Note that each virtual LED is listed using the same format as the
command used to configure it, except that the first letter is a lower-case
`l'.
\begin{tcolorbox}
\begin{Verbatim}[commandchars=\\\{\}]
\userinput{DL}
Saved LEDs
l,1,2,100
l,7,3,80
\end{Verbatim}
\end{tcolorbox}

The \texttt{XL} (eXecute LED) command can be used to manually turn an LED on
or off. This command has two fields: the virtual LED number and the desired
brightness level. The brightness level must be from 0 to 100, and is a
percentage of the maximum LED current. The commands shown below will turn
on virtual LED \#1 at 100\% of its possible brightness, and then turn it off.
\begin{tcolorbox}
\begin{Verbatim}[commandchars=\\\{\}]
\userinput{XL,1,100}
Executing LED 1
\userinput{XL,1,0}
Executing LED 1
\end{Verbatim}
\end{tcolorbox}


\subsection*{Flashes}
\subsection*{Patterns}
\subsection*{Random Pattern Sets}
\subsection*{Using the Keypad}
\end{document}


